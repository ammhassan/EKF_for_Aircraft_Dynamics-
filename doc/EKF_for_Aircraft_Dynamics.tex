\documentclass{article}

\usepackage{hyperref}
\usepackage{amsmath}
\usepackage{amssymb}
\usepackage{amsfonts}
\usepackage{amstext}

\title{\textbf{Extended Kalman Filter for Fixed-Wing Aircraft Dynamics}}
\author{\textbf{Ahmed M. Hassan}\thanks{am.hassan89@gmail.com}}

\begin{document}

\maketitle

\section{Introduction}
The goal of this project is to implement an extended Kalman filter (EKF) to estimate
a fixed-wing aicraft state vector from noisy sensor measurments. 
The first iteration of this project will be focused only on the longitudinal dynamics.
   
\section{Aircraft Model}
The nonlinear longitudinal dynamics for conventional fixed-wing aircraft can be written as follows \cite{Nelson, Stevens-Lewis}
\begin{equation}\label{Eq:Nonlinear_sys}
    \begin{split}
        \dot{U} &= -Q W - g \sin{\theta} + \frac{X}{m}\\
        \dot{W} &= Q U + g \cos{\theta} + \frac{Z}{m}\\
        \dot{Q} &= \frac{M}{I_{yy}}\\
        \dot{\theta} &= Q
    \end{split}
\end{equation}
The forces and moments can be broken down as follows
\begin{equation}
    \begin{split}
        X &= q S \biggl(C_X(\alpha) + \frac{\bar{c}}{2 V_T} C_{X_Q} Q + C_{X_{\delta_e}} \delta_e\biggr) +
        X_{t_0} + X_{\delta_t} \delta_t\\
        Z &= q S \biggl(C_Z(\alpha) + \frac{\bar{c}}{2 V_T} C_{Z_Q} Q + C_{Z_{\delta_e}} \delta_e\biggr)\\
        M &= q S \bar{c} \biggl(C_M(\alpha) + \frac{\bar{c}}{2 V_T} C_{M_Q} Q + C_{M_{\delta_e}} \delta_e \biggr)
    \end{split}
\end{equation}
Considering a trim condition in a cruise level flight, the nonlinear system \ref{Eq:Nonlinear_sys} 
can be further simplified to be on the following from
%This nonlinear model is very simple and has only very few nonlinear terms-need to have a more sophisticated model later
\begin{equation}\label{Eq:Nonlinear_sys_cruise}
    \begin{split}
        \dot{U} &= -Q W - g \cos{\theta_0} \Delta \theta+ X_U \Delta U + X_W \Delta W +
                 X_{\delta_e} \delta_e + X_{\delta_t} \delta_t\\
        \dot{W} &= Q U - g \sin{\theta_0} \Delta \theta + Z_U \Delta U + Z_W \Delta W + Z_{\delta_e} \delta_e\\
        \dot{Q} &= M_U \Delta U + M_W \Delta W + M_Q \Delta Q + M_{\delta_e} \delta_e + M_{\delta_t} \delta_t\\
        \dot{\theta} &= Q
    \end{split}
\end{equation}

The nonlinear system \ref{Eq:Nonlinear_sys} can be linearized and
written in a standard linear system form as follows \cite{Nelson, Stevens-Lewis}
\begin{equation} \label{Eq:Linearized_sys}
    \begin{bmatrix}
    \dot{U} \\
    \dot{W} \\
    \dot{Q} \\
    \dot{\theta} \\ 	
    \end{bmatrix}
    =
    \begin{bmatrix}
    X_u & X_w & 0 & -g \cos{\theta_0}\\
    Z_u & Z_w & U_0 & -g \sin{\theta_0}\\
    M_u & M_w & M_q & 0\\
    0 & 0 & 1 & 0 \\
    \end{bmatrix}
    \begin{bmatrix}
    U \\
    W \\
    Q \\
    \theta \\ 	
    \end{bmatrix}
    +
    \begin{bmatrix}
    X_{\delta_e} & X_{\delta_t} \\
    Z_{\delta_e} & 0 \\
    M_{\delta_e} & M_{\delta_t} \\
    0 & 0\\ 	
    \end{bmatrix}
    \begin{bmatrix}
    {\delta_e} \\
    {\delta_t}  	
    \end{bmatrix}
\end{equation}

In this project we will consider the longitudinal model of the aircraft "DELTA" given in \cite[PP. 561--563]{Mclean}
whose parameters are given as follows (at $U_0 = 75~ m/s$ and $\theta_0 = 2.7 ^\circ$)
\begin{equation}\label{Eq:DELTA_Params}
    \begin{split}
        m &= 300000 kg\\
        X_U &= -0.02\\
        X_W &= 0.1\\
        Z_U &= -0.23\\
        Z_W &= -0.634\\
        M_U &= -2.55*10^{-5}\\
        M_W &= -0.005\\
        M_Q &= -0.61\\
        X_{\delta_e} &= 0.14\\
        Z_{\delta_e} &= -2.9\\
        M_{\delta_e} &= -0.64\\
        X_{\delta_t} &= 1.56\\
        M_{\delta_t} &= 0.0054\\
    \end{split}
\end{equation}
where $\delta_t$ is considered to be from the trim thrust. 
As such, $\delta_t$ is allowed the between $1$ and $-0.56$ \cite{Hassan2016_JAST}.  

\section{Extended Kalman Filter Equations}

\section{Algorithm Structure}

\section*{References}
\bibliographystyle{plain}
\bibliography{ref}
\end{document}